% Created 2022-12-04 Sun 20:52
% Intended LaTeX compiler: pdflatex
\documentclass[11pt]{article}
\usepackage[utf8]{inputenc}
\usepackage[T1]{fontenc}
\usepackage{graphicx}
\usepackage{longtable}
\usepackage{wrapfig}
\usepackage{rotating}
\usepackage[normalem]{ulem}
\usepackage{amsmath}
\usepackage{amssymb}
\usepackage{capt-of}
\usepackage{hyperref}
\usepackage[margin=0.85in]{geometry}
\date{}
\title{Input and Ouput Exam Solutions}
\hypersetup{
 pdfauthor={Alexander Sepelenco},
 pdftitle={Input and Ouput Exam Solutions},
 pdfkeywords={},
 pdfsubject={},
 pdfcreator={Emacs 28.2 (Org mode 9.6)}, 
 pdflang={English}}
\begin{document}

\maketitle
\section{2021 Q3}
\label{sec:org077e20f}
\subsection{Part (b)}
\label{sec:org44abff3}
\begin{verbatim}
main :: IO ()
main = do
 emptyOutputFile -- reset SHOUT.log since we want a new file called SHOUT.log
 file1Contents <- readFile "files.txt"
 let listFiles = lines file1Contents
 if listFiles == []
     then return ()
     else do
         shoutIntoFile listFiles
 where emptyOutputFile = writeFile "SHOUT.log" ""

shoutIntoFile :: [FilePath] -> IO ()
shoutIntoFile [] = return ()
shoutIntoFile (xs:xss) = do
 fileContent <- readFile xs
 appendFile "SHOUT.log" . map (toUpper) $ fileContent
 shoutIntoFile xss
\end{verbatim}
\newpage
\section{2021 Q3}
\label{sec:org9623f5a}
\subsection{Part (c)}
\label{sec:org007737f}
\begin{verbatim}
main :: IO ()
main = do
 interleaves "input1.txt" "input2.txt"

interleaves :: FilePath -> FilePath -> IO ()
interleaves file1 file2 = do
 file1Contents <- readFile file1
 file2Contents <- readFile file2
 let linesOfFile1 = lines file1Contents
 let linesOfFile2 = lines file2Contents
 writeFile "output12.txt" . unlines $ interleaves' [] linesOfFile1 linesOfFile2

interleaves' :: [String] -> [String] -> [String] -> [String]
interleaves' zss xss [] = zss ++ xss
interleaves' zss [] yss = zss ++ yss
interleaves' zss (xs:xss) (ys:yss) = (interleaves' $! accumulator) xss yss
 where accumulator = zss ++ (xs : [ys])
\end{verbatim}
\section{2019 Q3}
\label{sec:orgbe95f44}
\subsection{Part (c)}
\label{sec:org588d7a6}
\begin{verbatim}
main :: IO ()
main = do
  putStr "Input a file with the form <root.ext>: "
  file <- getLine
  fileContents <- readFile file
  writeFile (outputFile file) . map (toLower) $ fileContents
    where outputFile file = takeWhile (/='.') file ++ ".log"
\end{verbatim}
\newpage
\section{2018 Q3}
\label{sec:org77b5ddf}
\subsection{Part (c)}
\label{sec:orgce3d31e}
\begin{verbatim}
toDOS :: FilePath -> FilePath
toDOS file = map (toUpper) (take 8 fileName) ++ map (toUpper) (take 4 fileExtension)
 where fileName      = takeWhile (/= '.') file
       fileExtension = dropWhile (/= '.') file -- includes dot therefore take 4 == .DAT
       eightFileName = take 8 fileName

\end{verbatim}
\subsection{Part (d)}
\label{sec:org4775c0f}
\begin{verbatim}
main :: IO ()
main = do
 putStr "Input a fileName with an extension: "
 file <- getLine
 fileContents <- readFile . toDOS $ file
 writeFile "LOWER.OUT" . map (toLower) $ fileContents
\end{verbatim}
\section{2015 Q3}
\label{sec:orga32876c}
\subsection{Part (c)}
\label{sec:orgd019baa}
\begin{verbatim}
hash :: String -> Int
hash str = (sum (map ord str)) `mod` 255

main :: IO ()
main = do
 putStr "Input the name of your file without an extension: "
 fileName <- getLine
 fileContents <- readFile . inputFile $ fileName
 writeFile (outputFile fileName) (show . hash $ fileContents)
  where inputFile fileName  = fileName ++ ".in"
        outputFile fileName = fileName ++ ".chk"
\end{verbatim}
\section{2014 Q4}
\label{sec:org2c48a00}
\subsection{Part (d)}
\label{sec:org54fa0f8}
\begin{verbatim}
main = do
 putStr "Input a filen of the form <Root>.<Extensions>: "
 file <- getLine
 let dosFile = toDOS file
 dosFileContents <- readFile dosFile
 writeFile ((take 8 dosFile) ++ ".OUT") . map (toLower) $ dosFileContents

toDOS :: FilePath -> FilePath
toDOS file = map (toUpper) dosNam ++ map (toUpper) dosExt
 where dosNam = take 8 $ takeWhile (/= '.') file
       dosExt = take 4 $ dropWhile (/= '.') file -- take 4 since . must be included
\end{verbatim}
\section{2013 Q4}
\label{sec:orge1054d1}
\subsection{Part (d)}
\label{sec:orgaacfc27}
\begin{verbatim}
main = do
 putStr "Input a filename without the extension: "
 file <- getLine
 fileContents <- readFile (file++".in")
 writeFile (file++".out") . map (toUpper) $ fileContents
\end{verbatim}
\end{document}